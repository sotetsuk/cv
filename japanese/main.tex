%
% Copyright (c) 1996, 2004, 2006, 2009, 2014, 2016
% Tama Communications Corporation. All rights reserved.
%
% Redistribution and use in source and binary forms, with or without
% modification, are permitted provided that the following conditions
% are met:
% 1. Redistributions of source code must retain the above copyright
%    notice, this list of conditions and the following disclaimer.
% 2. Redistributions in binary form must reproduce the above copyright
%    notice, this list of conditions and the following disclaimer in the
%    documentation and/or other materials provided with the distribution.
%
% THIS SOFTWARE IS PROVIDED BY THE AUTHOR AND CONTRIBUTORS ``AS IS'' AND
% ANY EXPRESS OR IMPLIED WARRANTIES, INCLUDING, BUT NOT LIMITED TO, THE
% IMPLIED WARRANTIES OF MERCHANTABILITY AND FITNESS FOR A PARTICULAR PURPOSE
% ARE DISCLAIMED.  IN NO EVENT SHALL THE AUTHOR OR CONTRIBUTORS BE LIABLE
% FOR ANY DIRECT, INDIRECT, INCIDENTAL, SPECIAL, EXEMPLARY, OR CONSEQUENTIAL
% DAMAGES (INCLUDING, BUT NOT LIMITED TO, PROCUREMENT OF SUBSTITUTE GOODS
% OR SERVICES; LOSS OF USE, DATA, OR PROFITS; OR BUSINESS INTERRUPTION)
% HOWEVER CAUSED AND ON ANY THEORY OF LIABILITY, WHETHER IN CONTRACT, STRICT
% LIABILITY, OR TORT (INCLUDING NEGLIGENCE OR OTHERWISE) ARISING IN ANY WAY
% OUT OF THE USE OF THIS SOFTWARE, EVEN IF ADVISED OF THE POSSIBILITY OF
% SUCH DAMAGE.
%
\documentclass{jarticle}
\usepackage{rireki}
%
% オプション
%
% 下記のオプションが利用可能です。
%
% \空行挿入		% 学歴と職歴の間に空行を挿入します
\begin{document}
%
% ID情報
%
\姓{\Large 小山田}
\名{\Large 創哲}
\姓読み{こやまだ}
\名読み{そうてつ}
\性別{男}					% 男|女
\誕生日{平成x年x月x日}
\年齢{xx}
%
% 顔写真
%
% 画像ファイルにはEPS フォーマット・縦横比4:3 のものをご使用ください。
% 縦を4cm に調整し、縦横比を変更せずに印刷します。
% 次のように指定します。
% \顔写真{photo.eps}
%
\顔写真{sotetsu_koyamada.pdf}
%
% 現住所
%
\現住所郵便番号{123}
\現住所{◯◯市◯◯町 1--2--3}
\現住所読み{まるまるし まるまるちょう}
\現住所市外局番{0123}
\現住所電話番号{45-6789}
\現住所呼び出し{◯◯ 方}
%
% 連絡先
%
\連絡先郵便番号{}
\連絡先{\tt xxx@gmail.com} % TODO
\連絡先読み{}
\連絡先市外局番{}
\連絡先電話番号{1234-56-7890} % TODO
\連絡先呼び出し{}
%
% 学歴、職歴
%
% 学歴、職歴を年月順に列挙してください。合計20個まで記入出来ます。
% 20個を超える部分は印刷されませんので、ご注意ください。
% 印刷順は、学歴=>職歴の順になります。
%
\学歴{平成16}{4}{埼玉県立春日部高等学校 入学}      % {年}{月}{内容}
\学歴{平成19}{3}{埼玉県立春日部高等学校 卒業}
\学歴{平成20}{4}{京都大学 経済学部 入学}
\学歴{平成25}{3}{京都大学 経済学部経済学科 卒業}
\学歴{平成25}{4}{京都大学大学院 情報学研究科システム科学専攻 修士課程 入学}
\学歴{平成27}{3}{京都大学大学院 情報学研究科システム科学専攻 修士課程 修了}
\学歴{平成27}{4}{京都大学大学院 情報学研究科システム科学専攻 博士課程 入学}
\学歴{平成28}{3}{京都大学大学院 情報学研究科システム科学専攻 博士課程 在学中}
\職歴{平成27}{4}{株式会社リクルートホールディングス 入社}
\職歴{平成27}{7}{株式会社リクルートテクノロジーズ 出向}
\職歴{平成28}{3}{現在に至る}
%
% 資格
%
% 資格を取得年月順に列挙してください。9つまで記入できます。
% 9つを超える部分は印刷されませんので、ご注意ください。
%
% \資格{平成xx}{xx}{普通自動車一種免許}            % {取得年}{取得月}{資格}
%
% 個人情報
%
% 志望の動機と本人希望記入欄はlatex のコマンドを記述できます。
%
\志望の動機{
	\begin{tabular}{ll}
	{\gt 志望の動機} & ◯◯◯◯◯◯◯◯◯◯◯◯◯◯◯◯◯\\
	{\gt 特技} & ◯◯◯\\
	{\gt 好きな学科} & ◯◯◯\\
	{\gt アピールポイント} & ◯◯◯◯◯◯◯◯◯◯◯◯◯◯◯\\
	\end{tabular}
}
\本人希望記入欄{
	私が希望する仕事の条件は下記の通りです。
	\begin{itemize}
	\item ◯◯◯◯◯◯◯◯◯◯◯◯◯◯◯◯◯
	\item ◯◯◯◯◯◯◯◯◯◯◯◯◯◯◯◯◯
	\item ◯◯◯◯◯◯◯◯◯◯◯◯◯◯◯◯◯
	\end{itemize}
}
%
% その他
%
\通勤時間{約 30分}
\扶養家族数{0}					% 人数(配偶者を除きます)
\配偶者{なし}					% あり|なし
\配偶者の扶養義務{なし}				% あり|なし

\サイン{Your Signature}

\end{document}
