%%%%%%%%%%%%%%%%%%%%%%%%%%%%%%%%%%%%%%%%%
% Medium Length Graduate Curriculum Vitae
% LaTeX Template
% Version 1.1 (9/12/12)
%
% This template has been downloaded from:
% http://www.LaTeXTemplates.com
%
% Original author:
% Rensselaer Polytechnic Institute (http://www.rpi.edu/dept/arc/training/latex/resumes/)
%
% Important note:
% This template requires the res.cls file to be in the same directory as the
% .tex file. The res.cls file provides the resume style used for structuring the
% document.
%
%%%%%%%%%%%%%%%%%%%%%%%%%%%%%%%%%%%%%%%%%

%----------------------------------------------------------------------------------------
%	PACKAGES AND OTHER DOCUMENT CONFIGURATIONS
%----------------------------------------------------------------------------------------

\documentclass[margin, 10pt]{res} % Use the res.cls style, the font size can be changed to 11pt or 12pt here

\usepackage{helvet} % Default font is the helvetica postscript font
%\usepackage{newcent} % To change the default font to the new century schoolbook postscript font uncomment this line and comment the one above

\setlength{\textwidth}{5.1in} % Text width of the document
\usepackage{url}

\begin{document}

%----------------------------------------------------------------------------------------
%	NAME AND ADDRESS SECTION
%----------------------------------------------------------------------------------------

\moveleft.5\hoffset\centerline{\large\bf Sotetsu Koyamada} % Your name at the top
 
\moveleft\hoffset\vbox{\hrule width\resumewidth height 1pt}\smallskip % Horizontal line after name; adjust line thickness by changing the '1pt'
 
\moveleft.5\hoffset\centerline{Graduate School of Informatics, Kyoto University} % Your address
\moveleft.5\hoffset\centerline{Yoshidahonmachi 36-1, sakyo-ku, Kyoto-city, Kyoto, Japan. 606-8501}
\moveleft.5\hoffset\centerline{\url{koyamada-s@sys.i.kyoto-u.ac.jp}}
\moveleft.5\hoffset\centerline{\url{https://sotets.uk}}

%----------------------------------------------------------------------------------------

\begin{resume}

\section{{\small RESEARCH\\INTERESTS}}

My primary research interest is reinforcement learning, and I am particularly interested in both theoretical and practical interface between its algorithms and other fields of machine learning. I am also interested in neural networks, natural language processing, and sensitivity analysis in general as well.

\section{{\small EDUCATION}}

{\sl {\bf Ph.D. candidate of Informatics}} \hfill {\small \underline{Apr 2015 - Present}} \\
{\it Kyoto University} \\
Advisor: Shin Ishii

{\sl {\bf Master of Informatics}} \hfill {\small \underline{Apr 2013 - Mar 2015}}  \\
{\it Kyoto University} \\
Advisor: Shin Ishii \\
Thesis title: ``Principal Sensitivity Analysis and Its Application to Knowledge Discovery in Functional Neuroimaging''

{\sl {\bf Bachelor of Economics}} \hfill {\small \underline{Apr 2008 - Mar 2013}} \\
{\it Kyoto University} \\
Advisor: Masaaki Iiyama
  
\section{{\small PROFESSIONAL\\EXPERIENCE}}

{\sl {\bf Research internship}} \hfill {\small \underline{Aug 2016 - Present}} \\
{\it National Institute of Advanced Industrial Science and Technology, Japan} \\
Developed a new training objective function for neural sequence prediction, which generalizes the maximum likelihood-based and reinforcement learning-based objective functions using $\alpha$-divergence.

{\sl {\bf Machine learning engineer}} \hfill {\small \underline{Apr 2015 - Present}}\\
{\it Recruit Holdings Co., Ltd., Japan} \\
Constructed predictive APIs on Hadoop and Spark platform to improve KPI (key performance indicator) performances for more than 30 web services.

{\sl {\bf Research internship}} \hfill {\small \underline{Oct 2013 - Mar 2015}} \\
{\it ATR Cognitive Mechanisms Laboratories, Japan}  \\
Developed a subject-independent brain decoder using neural networks and proposed a new algorithm for data-driven scientific discovery from nonlinear classifiers.

\section{{\small RESEARCH} }
{\sl {\bf Books}} \vspace{0.5em}
\begin{itemize}
\item {\bf S. Koyamada} et al.: Japanese translation of ``Algorithms for Reinforcement Learning'' by C. Szepesv{\'a}ri, Kyoritsu Shuppan.
  \begin{itemize}
  \item Chief editor. Managed the entire project ran by 12 team members.
  \item Wrote an additional chapter about deep reinforcement learning.
  \end{itemize}
\end{itemize}

{\sl {\bf Publications (refereed)}} \vspace{0.5em}
\begin{itemize}
\item {\bf S. Koyamada}, Y. Kikuchi, A. Kanemura, S. Maeda, and S. Ishii: ``Neural sequence model training via $\alpha$-divergence minimization.'' ICML Workshop on Learning to Generate Natural Language, 2017.
\item {\bf S. Koyamada}, M. Koyama, K. Nakae, and S. Ishii: ``Principal sensitivity analysis.'' In Proceedings of the Pacific-Asia Conference on Knowledge Discovery and Data Mining (PAKDD), 621-632, 2015.
\item {\bf S. Koyamada}, Y. Shikauchi, K. Nakae, and S. Ishii: ``Construction of subject independent brain decoders for human fMRI with deep learning.'' The International Conference on Data Mining, Internet Computing, and Big Data, 60-68, 2014.
\end{itemize}

{\sl {\bf Pre-prints (not refereed)}} \vspace{0.5em}
\begin{itemize}
\item {\bf S. Koyamada}, Y. Shikauchi, K. Nakae, M. Koyama, S. Ishii ``Deep learning of fMRI big data: a novel approach to subject-transfer decoding.'' arXiv:1502.00093, 2015.
\end{itemize}

{\sl {\bf Other presentations (not refereed)}} \vspace{0.5em}
\begin{itemize}
\item {\bf S. Koyamada}: ``Principal Sensitivity Analysis.'' Machine Learning Summer School 2015 Kyoto, Kyoto, Sep 1, 2015 (poster presentation)
\item {\bf S. Koyamada}, Y. Shikauchi, K. Nakae, M. Koyama, and S. Ishii: ``Knowledge Discovery for Nonlinear Classifier in Functional Neuroimaging.'' 10th AEARU Workshop on Computer Science and Web Technology, Tsukuba, Feb 26, 2015 (poster presentation)
\item {\bf S. Koyamada}, Y. Shikauchi, K. Nakae, and S. Ishii: ``Learning the subject-independent discriminative features from the large-scale fMRI database.'' Neuro2014, Yokohama, Sep 13, 2014 (poster presentation)
\end{itemize}

\section{{\small GRANTS AND\\SCHOLARSHIPS}}
{\sl {\bf Student Scholarship}} \hfill {\small \underline{Apr 2013 - Mar 2015}} \\
Japan Student Services Organization (JASSO), Japan \\
Approx. 1,056,000 yen

\section{{\small TEACHING}}
{\sl {\bf Teaching assistant}} \hfill {\small \underline{Jul 23, 2014}} \\
Lecture sessions on deep learning, Kyoto University, Japan

{\sl {\bf Teaching assistant}} \hfill {\small \underline{Oct 2013 - Mar 2014}} \\
``Introduction to Computer Science.'' Kyoto University, Japan

\section{{\small SKILLS}}
{\sl {\bf Programming skills}} \vspace{0.5em}
\begin{itemize}
\item Programming language: Python, Go, C++, Java, R
  \begin{itemize}
  \item GitHub repo: https://github.com/sotetsuk
  \end{itemize}
\item Deep learning framework: PyTorch, Chainer, TensorFlow
\item Middleware/Infrastructure: Hadoop, Spark, RDBMS, AWS, GCP, Docker
\item Other tools: Git, SQL, LaTeX
\end{itemize}

{\sl {\bf Language}}\\
Japanese (native), English

\end{resume}
\end{document}
