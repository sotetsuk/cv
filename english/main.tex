%%%%%%%%%%%%%%%%%%%%%%%%%%%%%%%%%%%%%%%%%
% Medium Length Graduate Curriculum Vitae
% LaTeX Template
% Version 1.1 (9/12/12)
%
% This template has been downloaded from:
% http://www.LaTeXTemplates.com
%
% Original author:
% Rensselaer Polytechnic Institute (http://www.rpi.edu/dept/arc/training/latex/resumes/)
%
% Important note:
% This template requires the res.cls file to be in the same directory as the
% .tex file. The res.cls file provides the resume style used for structuring the
% document.
%
%%%%%%%%%%%%%%%%%%%%%%%%%%%%%%%%%%%%%%%%%

%----------------------------------------------------------------------------------------
%	PACKAGES AND OTHER DOCUMENT CONFIGURATIONS
%----------------------------------------------------------------------------------------

\documentclass[margin, 10pt]{res} % Use the res.cls style, the font size can be changed to 11pt or 12pt here

\usepackage{helvet} % Default font is the helvetica postscript font
%\usepackage{newcent} % To change the default font to the new century schoolbook postscript font uncomment this line and comment the one above

\setlength{\textwidth}{5.1in} % Text width of the document
\usepackage{url}

\begin{document}

%----------------------------------------------------------------------------------------
%	NAME AND ADDRESS SECTION
%----------------------------------------------------------------------------------------

\moveleft.5\hoffset\centerline{\large\bf Sotetsu Koyamada} % Your name at the top
 
\moveleft\hoffset\vbox{\hrule width\resumewidth height 1pt}\smallskip % Horizontal line after name; adjust line thickness by changing the '1pt'
 
\moveleft.5\hoffset\centerline{Graduate School of Informatics, Kyoto University} % Your address
\moveleft.5\hoffset\centerline{Yoshidahonmachi 36-1, sakyo-ku, Kyoto-city, Kyoto, Japan. 606-8501}
\moveleft.5\hoffset\centerline{\url{koyamada-s@sys.i.kyoto-u.ac.jp}}
\moveleft.5\hoffset\centerline{\url{https://sotets.uk}}

%----------------------------------------------------------------------------------------

\begin{resume}

\section{{\small INTERESTS}}
I am interested in developing machine learning solutions for familiar problems to us by utilizing both of the engineering approach and scientific method.
As a research topic, my primary interest is reinforcement learning, and its application to imperfect information game (Mahjong, Poker and so on).

% \section{{\small RESEARCH\\INTERESTS}}
% My primary research interest is reinforcement learning, and I am particularly interested in both theoretical and practical interface between its algorithms and other fields of machine learning. I am also interested in neural networks, natural language processing, and sensitivity analysis in general as well.

\section{{\small EDUCATION}}

{\sl {\bf Ph.D. candidate of Informatics}} \hfill {\small \underline{Apr 2015 - Present}} \\
{\it Kyoto University} \\
Advisor: Shin Ishii

{\sl {\bf Master of Informatics}} \hfill {\small \underline{Apr 2013 - Mar 2015}}  \\
{\it Kyoto University} \\
Advisor: Shin Ishii \\
Thesis title: ``Principal Sensitivity Analysis and Its Application to Knowledge Discovery in Functional Neuroimaging''

{\sl {\bf Bachelor of Economics}} \hfill {\small \underline{Apr 2008 - Mar 2013}} \\
{\it Kyoto University} \\
Advisor: Masaaki Iiyama
  
\section{{\small PROFESSIONAL\\EXPERIENCE}}

{\sl {\bf Research intern}} \hfill {\small \underline{Apr 2018 - Mar 2020}} \\
{\it Microsoft Research Asia, Beijing, China} \\
Developed the first super-human Mahjong AI, Suphx, with the other team members.
I developed a distributed reinforcement learning system, which can efficiently work over a hundred GPUs, resulting in more than $20 \times$ speeding up.
Utilizing the RL system, I run tremendeous number of experiments and improved the AI performance dramatically ($2.6 \times$ stronger). 
Also, I contributed to the domain knowledge of Japanese Mahjong as an only Japanese member.

{\sl {\bf Research assistant}} \hfill {\small \underline{Aug 2016 - Mar 2018}} \\
{\it National Institute of Advanced Industrial Science and Technology, Tokyo, Japan} \\
Developed a new training objective function for neural sequence prediction, which uses $\alpha$-divergence to theoretically bridge the gap between maximum likelihood-based methods and reinforcement learning.

{\sl {\bf Machine learning engineer}} \hfill {\small \underline{Apr 2015 - Mar 2018}}\\
{\it Recruit Holdings Co., Ltd., Tokyo, Japan} \\
Constructed predictive APIs on Hadoop and Spark platform to improve KPIs (key performance indicators). The system was working for more than 30 major web services, including suumo.jp, jalan.net, and carsensor.net.

{\sl {\bf Research assistant}} \hfill {\small \underline{Oct 2013 - Mar 2015}} \\
{\it ATR Cognitive Mechanisms Laboratories, Kyoto, Japan}  \\
Developed a subject-independent brain decoder using neural networks and proposed a new algorithm for data-driven scientific discovery from nonlinear classifiers.

\section{{\small RESEARCH} }
{\sl {\bf Books}} \vspace{0.5em}
\begin{itemize}
\item {\bf S. Koyamada} et al.: Japanese translation of ``Algorithms for Reinforcement Learning'' by C. Szepesv{\'a}ri, Kyoritsu Shuppan.
  \begin{itemize}
  \item Chief editor. Managed the entire project ran by 12 team members.
  \item Wrote an additional chapter about deep reinforcement learning.
  \end{itemize}
\end{itemize}

{\sl {\bf Publications}} \vspace{0.5em}
\begin{itemize}
\item J. Li, {\bf S. Koyamada}, Q. Yi, G. Liu, C. Wang, R. Yang, L. Zhao, T. Qin, T.Y. Liu, H.W. Hon: {\bf ``Suphx: Mastering Mahjong with Deep Reinforcement Learning.''} arxiv:2003.13590, 2020.
\item {\bf S. Koyamada}, Y. Kikuchi, A. Kanemura, S. Maeda, and S. Ishii: {\bf ``Neural sequence model training via $\alpha$-divergence minimization.''} ICML Workshop on Learning to Generate Natural Language, 2017.
\item {\bf S. Koyamada}, M. Koyama, K. Nakae, and S. Ishii: {\bf ``Principal sensitivity analysis.''} In Proceedings of the Pacific-Asia Conference on Knowledge Discovery and Data Mining (PAKDD), 621-632, 2015.
\item {\bf S. Koyamada}, Y. Shikauchi, K. Nakae, M. Koyama, S. Ishii {\bf ``Deep learning of fMRI big data: a novel approach to subject-transfer decoding.''} arXiv:1502.00093, 2015.
\item {\bf S. Koyamada}, Y. Shikauchi, K. Nakae, and S. Ishii: {\bf ``Construction of subject independent brain decoders for human fMRI with deep learning.''} The International Conference on Data Mining, Internet Computing, and Big Data, 60-68, 2014.
\end{itemize}

% {\sl {\bf Pre-prints (not refereed)}} \vspace{0.5em}
% \begin{itemize}
% \item {\bf S. Koyamada}, Y. Shikauchi, K. Nakae, M. Koyama, S. Ishii ``Deep learning of fMRI big data: a novel approach to subject-transfer decoding.'' arXiv:1502.00093, 2015.
% \end{itemize}

% {\sl {\bf Other presentations (not refereed)}} \vspace{0.5em}
{\sl {\bf Other presentations}} \vspace{0.5em}
\begin{itemize}
\item S. Koyamada: ``Principal Sensitivity Analysis.'' Machine Learning Summer School 2015 Kyoto, Kyoto, Sep 1, 2015 (poster presentation)
\item S. Koyamada, Y. Shikauchi, K. Nakae, M. Koyama, and S. Ishii: ``Knowledge Discovery for Nonlinear Classifier in Functional Neuroimaging.'' 10th AEARU Workshop on Computer Science and Web Technology, Tsukuba, Feb 26, 2015 (poster presentation)
\item S. Koyamada, Y. Shikauchi, K. Nakae, and S. Ishii: ``Learning the subject-independent discriminative features from the large-scale fMRI database.'' Neuro2014, Yokohama, Sep 13, 2014 (poster presentation)
\end{itemize}

\section{{\small GRANTS AND\\SCHOLARSHIPS}}
{\sl {\bf Repayment Exemption of Student Loan for Students with Excellent Grades}} \hfill {\small \underline{Apr 2013 - Mar 2015}} \\
Japan Student Services Organization (JASSO), Japan \\
Approx. 1,056,000 JPY

{\sl {\bf Student Scholarship}} \hfill {\small \underline{Apr 2018 -Sep 2018}} \\
Tobitate! Study Abroad Initiative, Japan \\
Approx. 870,000 JPY

\section{{\small TEACHING}}
{\sl {\bf Teaching assistant}} \hfill {\small \underline{Jul 23, 2014}} \\
Lecture sessions on deep learning, Kyoto University, Japan

{\sl {\bf Teaching assistant}} \hfill {\small \underline{Oct 2013 - Mar 2014}} \\
``Introduction to Computer Science.'' Kyoto University, Japan

\section{{\small SKILLS}}
{\sl {\bf Programming skills}} \vspace{0.5em}
\begin{itemize}
\item Programming language: {\bf C++}, {\bf Python}, C\#, Go, Java, R
  \begin{itemize}
  \item GitHub repo: https://github.com/sotetsuk
  \end{itemize}
\item Deep learning framework: PyTorch, TensorFlow
\item Middleware/Infrastructure: Docker, AWS, GCP, RDBMS, Hadoop, Spark
\item Others: Git, SQL, LaTeX
\end{itemize}

{\sl {\bf Language}}\\
Japanese (native), English

\end{resume}
\end{document}
